\chapter{Introdução}

%=====================================================

% A introdução geral do documento pode ser apresentada através das seguintes seções: Desafio, Motivação, Proposta, Contribuição e Organização do documento (especificando o que será tratado em cada um dos capítulos). O Capítulo 1 não contém subseções\footnote{Ver o Capítulo \ref{cap-exemplos} para comentários e exemplos de subseções.}.
O problema sugerido pelo orientador consiste em modelar e implementar uma solução para o problema de envio ordenado. Existe uma certa quantidade de produtos disponíveis e esses produtos precisam ser enviados em uma quantidade finita de caminhões. O objetivo do trabalho, é encontrar uma solução que minimize a quantidade de caminhões necessários para realizar o transporte de todas as mercadorias. Os items serão que transportados, podem ser separados em pedaços menores de forma que seja possível separar os pedaços em caminhões diferentes. 

\section{LIMITAÇÕES E DETALHES}
  Antes de começar a análise e modelagem do problema, vou definir alguns detalhes, limitações e problemas encontrados. 
  \begin{itemize}
    \item O custo mínimo é calculado sobre a quantidade minima de viagens necessárias, de forma que eu consiga fazer o transporte de todos os produtos. 
    \item Cada caminhão possui um limite de peso que ele consegue carregar. E esse peso precisa ser respeitado pelos itens que estarão dentro desse caminhão. 
    \item Os caminhões podem carregar diferentes itens, contanto que o peso seja respeitado. 
    \item Na modelagem do problema, não foi considerado os pares ordenados. Isso porque não foi possível chegar em uma forma de ter pares ordenados utilizando programação linear.
    \item Se necessário, um item pode ser separado em diferentes caminhões.
  \end{itemize}

\section{VARIÁVEIS DO PROBLEMA}
  Nessa sessão é apresentado as váriaveis e restrições que foram realizados na usados na modelagem do problema.

\begin{itemize}
  \item n = Quantidade de itens que terão que ser transportados.
  \item C = Capacidade total possível para cada caminhão.
  \item {\[ x_{i} \] = \[x_{i}\] representa o item.}
  \item \[ x_{i,j}\] = \[x_{i,j}\] representa o item i no caminhão j.
\end{itemize}


\begin{itemize}

\item \verb#defesa#: é gerado um documento em espaço 1,5, frente simples e sem as páginas iniciais adicionais; é uma versão adequada para receber as anotações dos membros da banca de defesa.

\item \verb#final#: é gerado um documento em espaço simples, frente/verso, com páginas iniciais (capa, ficha catalográfica, folha de aprovação, agradecimentos, etc). É uma versão bem mais compacta, mais ecológica e ideal para a impressão definitiva.

\end{itemize}

Para obter os melhores resultados, compile este modelo usando a seguinte sequência de passos:

\begin{quote}
\begin{footnotesize}
\begin{verbatim}
pdflatex  main          // compilação inicial
bibtex main             // processa referências bibliográficas
pdflatex  main          // compilação final
\end{verbatim}
\end{footnotesize}
\end{quote}

ou

\begin{quote}
\begin{footnotesize}
\begin{verbatim}
make                    // faz tudo...
\end{verbatim}
\end{footnotesize}
\end{quote}

Os principais itens considerados na formatação deste documento foram:

\begin{itemize}

\item Papel em formato A4, com margens de 20 mm à direita e embaixo, 30 mm nos demais lados. Não devem ser usados cabeçalhos ou rodapés além dos que estão aqui propostos.

\item O texto principal do documento escrito em 12 pontos. O fonte principal do texto pode ser selecionado no arquivo \verb#packages.tex#.

\item Código-fonte, listagens e textos similares são formatados em fonte Courier 12 ou 10 pontos.

\item O espaçamento padrão entre linhas é 1,5 linhas (1 linha na versão final). Não inserir espaços adicionais entre parágrafos normais. Figuras, tabelas, listagens e listas de itens devem ter um espaço adicional antes e após os mesmos.

\item As páginas iniciais não são numeradas.

\item O corpo do texto é numerado com algarismos arábicos (1, 2, 3, ...) a partir da introdução, ate o final do documento. Os números de página devem estar situados no alto à direita (páginas direitas) ou à esquerda (páginas esquerdas).

\item Expressões em inglês, grego, latim ou outras línguas devem ser enfatizadas em itálico, como \emph{sui generis} ou \emph{scheduling} (use o comando \verb#\emph{...}#).

\item Para reforçar algo, deve-se usar somente \textbf{negrito}. \underline{Sublinhado} ou MAIÚSCULAS não devem ser usados como forma de ênfase!

\item As notas de rodapé também têm um modelo\footnote{As notas de rodapé dever ser escritas em tamanho 10 pt, numeradas em arábico.}. Notas de rodapé servem para fazer algum comentário paralelo; não as use para colocar URLs, referências bibliográficas ou significado de siglas.

\end{itemize}

Felizmente o \LaTeX\ resolve a maior parte dessas questões!

%=====================================================
